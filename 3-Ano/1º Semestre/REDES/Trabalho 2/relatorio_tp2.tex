\documentclass{llncs}
\usepackage{times}
\usepackage[T1]{fontenc}
\usepackage[portuges]{babel}
\usepackage[utf8]{inputenc}
\usepackage{aeguill}


% Comentar para not MAC Users
\usepackage[applemac]{inputenc}

\usepackage{a4}
%\usepackage[margin=3cm,nohead]{geometry}
\usepackage{epstopdf}
\usepackage{graphicx}
\usepackage{fancyvrb}
\usepackage{amsmath}
%\renewcommand{\baselinestretch}{1.5}

\begin{document}
\mainmatter
\title{TP2: Protocolo IPv4}

\titlerunning{TP2: Protocolo IPv4}

\author{Ana Luísa Pereira César\and Hugo Fernandes Matias \and João Nuno Cardoso Gonçalves de Abreu }

\authorrunning{Ana Luísa Pereira de César \and Hugo Fernandes Matias \and João Nuno Cardoso Gonçalves de Abreu }


\institute{
Universidade do Minho, Departamento de  Informatica, 4710-057 Braga, Portugal\\
e-mail: \{a86038,a85370,a84802\}@alunos.uminho.pt
}

\date{}
\bibliographystyle{splncs}

\maketitle

\section{Respostas}

\subsection{Active o wireshark ou o tcpdump no pc s1. Numa shell de s1, execute o comando traceroute -I para o endereço IP  do  host h5.    }
temos que colocar as imagens aqui
\subsection{Registe e analise o tráfego ICMP enviado por s1 e o tráfego ICMP recebido como resposta. Comente os resultados face ao  comportamento esperado.}

Inicialmente, o host s1 envia 3 segmentos UDP cujo destino é o host h5 e com TTL=1, não obtendo resposta. \\
Recebe, posteriormente uma resposta do router r2 a dizer que o TTL foi excedido. Isto acontece porque o TTL sendo 1 não é suficiente para alcançar o destino pretendido (h5). Com este TTL faz apenas um salto, sendo este o motivo da resposta vir do router r2.


\subsection{Qual deve ser o valor inicial mínimo do campo TTL para alcançar o destino   h5? Verifique na prática que a sua resposta está correta}



\subsection{Qual o valor médio do tempo de ida-e-volta (Round-Trip Time) obtido?}

\end{document}